
%%%%%%%%%%%%%%%%%%%%%%%%%%%%%%%%%%%%%%%%%%%%%%%%%%%%%%%%%%%%%%%%%%
% Sample template for MIT Junior Lab Student Written Summaries
% Available from http://web.mit.edu/8.13/www/Samplepaper/sample-paper.tex
% Last Updated April 12, 2007
% Adapted from the American Physical Societies REVTeK-4 Pages
% at http://publish.aps.org
%\newcommand{\mjmlegacy}{}

\setlength{\paperheight}{11in}
% http://tex.stackexchange.com/questions/74636/mla-package-and-thumbpdf
\makeatletter
\@namedef{ver@thumbpdf.sty}{}
\makeatother
%\documentclass[aps,secnumarabic,balancelastpage,amsmath,amssymb,nofootinbib]{revtex4}
\input{myrevtexheaders.tex}
\usepackage[section]{placeins}


\input{myskeletonpackages.tex}

\newcolumntype{.}[1]{D{.}{.}{#1}}
%\usepackage[maxfloats=30]{morefloats}   %mjm   saving up figs for the end   
% no param on old version stuck at 36
\usepackage{morefloats}        %mjm   saving up figs for the end   
\usepackage{graphicx}        %mjm   saving up figs for the end   

% will need modificaitons 
%mjmlegacy
%\input{recent_template.tex}



% https://tex.stackexchange.com/questions/121601/automatically-wrap-the-text-in-verbatim
\usepackage{listings}
\lstset{
basicstyle=\small\ttfamily,
columns=flexible,
breaklines=true
}

% none of this fking fking works for a f F 
%mjmlegacy
%\newcommand{\mjmverbatim}{lstlisting} \newcommand{\mjmbeginverbatim}{\begin{lstlisting}} \newcommand{\mjmendverbatim}{\end{lstlisting}} \newcommand{\mjmmangle}[1]{keep/#1}



%# CHANGE VERSION AND STATUS MANUALLY 
% need a draft/notes/release flag

% https://tex.stackexchange.com/questions/5894/latex-conditional-expression
%At the command-line, you can do \def\MYFLAG{} and then test if \MYFLAG is defined in your document (or an included style file) with \ifdefined\MYFLAG ... \else ... \fi.
% needs trailing space for the sample bibtex doh
% leading spaces mess up the entry thought 
\def\xxmjmrelease{0.10 }
\ifdefined\mjmrelease
\newcommand{\mjmstatus}{ PUBLIC NOTES }
\newcommand{\mjmversion}{\mjmrelease} %%%%%%%%%%%%%
\newcommand{\mjmtrno}{MJM-2026-001}
\newcommand{\mjmbib}{\mjmtrno-\mjmversion}
\newcommand{\mjmstatuswarn}{{\bf{  }}   }
% 2021-09-29 wanted version wth for brownie
%\newcommand{\mjmbib}{\mjmtrno-\mjmversion-\mjmrelease}
%\newcommand{\mjmbib}{\mjmtrno}
\else
\newcommand{\mjmstatus}{ NOT public NOTES }
\newcommand{\mjmversion}{0.00} %%%%%%%%%%%%%
\newcommand{\mjmtrno}{MJM-2026-001}
%\newcommand{\mjmbib}{\mjmtrno-\mjmversion}
\newcommand{\mjmbib}{\mjmtrno}
%\newcommand{\mjmstatuswarn}{  }
\newcommand{\mjmstatuswarn}{{\bf{This document is a non-public DRAFT and contents may be speculative or undocumented or simple musings and should be read as such.  }}   }
\fi

%\newcommand{\mjmstatus}{ NOT public NOTES }



\newcommand{\mjmtitle}{Alzheimer's Disease : 10k maniacs agree its an elephant}
\begin{comment}
\newcommand{\xxmjmmakedate}{ }
\newcommand{\xxmjmauthor}{Mike J Marchywka }
\newcommand{\xxmjmbasename}{\jobname}
\newcommand{\xxmjmaddbio}{mjm_tr,releases}
\newcommand{\xxmjmversion}{ 0.00 }
\newcommand{\xxmjmtrno}{MJM-2018-009}
\newcommand{\xxmjmbibday}{12}
\newcommand{\xxmjmbibmo}{12}
\newcommand{\xxmjmbibyear}{2018}
\newcommand{\xxmjmemail}{marchywka@hotmail.com}
\end{comment}

\input{mycommands.tex}


\newcommand{\mjmauthor}{Mike J Marchywka }
\newcommand{\mjmmakedate}{2026-01-17 }
\newcommand{\mjmbasename}{\jobname}
\newcommand{\mjmaddbio}{mjm_tr,releases}
%\newcommand{\mjmversion}{0.00}
%\newcommand{\mjmtrno}{MJM-2026-001}
%\newcommand{\mjmbibday}{17}
%\newcommand{\mjmbibmo}{01}
%\newcommand{\mjmbibyear}{2026}

% the build script changes these to creation day doh 
\newcommand{\mjmbibday}{17}
\newcommand{\mjmbibmo}{01}
\newcommand{\mjmbibyear}{2026}


\newcommand{\mjmmakebibday}{\number\day}
\newcommand{\mjmmakebibmo}{\number\month}
\newcommand{\mjmmakebibyear}{\number\year}

\newcommand{\mjmbibtype}{techreport}

\newcommand{\mjmbibname}{marchywka-\mjmbib}
\mjmstartbib{\mjmbibtype}{\mjmbibname}


\newcommand{\mjmemail}{marchywka@hotmail.com}
%\newcommand{\mjmaddr}{157 Zachary Dr Talking Rock GA 30175 USA}
\newcommand{\mjmaddr}{157 Zachary Dr Talking Rock GA 30175 USA}
\mjmaddbib{title}{\mjmtitle}
\mjmaddbib{author}{\mjmauthor}
\mjmaddbib{type}{\mjmbibtype}
%\mjmaddbib{name}{marchywka-\mjmbib}
\mjmaddbib{name}{\mjmbibname}
\mjmaddbib{number}{\mjmtrno}
\mjmaddbib{version}{\mjmversion}
\mjmaddbib{institution}{not institutionalized, independent }
\mjmaddbib{address}{ \mjmaddr}
\mjmaddbib{date}{\today}
\mjmaddbib{startdate}{\mjmbibyear -\mjmbibmo -\mjmbibday }
%\mjmaddbib{day}{\mjmbibday}
%\mjmaddbib{month}{\mjmbibmo}
%\mjmaddbib{year}{\mjmbibyear}
\mjmaddbib{day}{\mjmmakebibday}
\mjmaddbib{month}{\mjmmakebibmo}
\mjmaddbib{year}{\mjmmakebibyear}

\mjmaddbib{author1email}{\mjmemail}
\mjmaddbib{contact}{\mjmemail}
\mjmaddbib{author1id}{orcid.org/0000-0001-9237-455X}
\CatchFileEdef\mjmpages{\mjmbasename.last_page}{\endlinechar=-1\relax}
% TODO FIXME add this to the skeleton text 
%\mjmaddbib{pages}{ \input{\mjmbasename.last_page}}
\mjmaddbib{pages}{ \mjmpages}
%\mjmaddbib{filename}{\mjmbasename}

\begin{comment}
\newcommand{\checkfortwo}[1]{%
  \ifcsname#1\endcsname%
  VERSION =        \{\mjmversion \today \mjmstatus \},
  \else%
  VERSION =        \{\mjmversion\},
  \fi%
}

\end{comment}
%\mjmaddbib{bibtex}{\mjmfullbib}

\mjmdonebib



\lhead{\mjmauthor,  \mjmtrno }


%\lhead{M Marchywka,  \mjmtrno }
%\rhead{ \mjmversion not for public release}
%\rhead{ { \today }  v. \mjmversion for release without review }
%\rhead{ { \today }  v. \mjmversion NOT public DRAFT }
%\rhead{ { \today }  v. \mjmversion { }  NOT public NOTES }
\rhead{ { \today }  v. \mjmversion { }  \mjmstatus }

\newfloatcommand{capbtabbox}{table}[][\FBwidth]


%%%% build flags 
%\newlength{\desttabw}  \setlength{\desttabw}{4in}
\newlength{\desttabw}  \setlength{\desttabw}{\textwidth}
\newlength{\chainwidth}  \setlength{\chainwidth}{.4\textwidth }
\newlength{\slantwidth}  \setlength{\slantwidth}{.2\textwidth }
\newlength{\subfigwidth}  \setlength{\subfigwidth}{.3\textwidth }
\newlength{\fullfigwidth}  \setlength{\fullfigwidth}{.8\textwidth }
\newlength{\subwfigwidth}  \setlength{\subwfigwidth}{.75\textwidth }
\newlength{\subwfigwidthrot}  \setlength{\subwfigwidthrot}{\textwidth }
\newlength{\myboxwidth}  \setlength{\myboxwidth}{.3\textwidth }
\newlength{\picwidth}  \setlength{\picwidth}{.4\textwidth }
% set to center for nowmal output 
\newcommand{\destflushtab}{flushleft}
\includecomment{mdpicomment}
\excludecomment{draftcomment}
\excludecomment{badmathcomment}
\excludecomment{showworkcomment}
% this does not fing work ... 

\newcommand{\mjmed}[1]{
%\begin{mjmedx} 
[ mjm : #1   ]
%\end{mjmedx} 
}  

% thinking outload
\newcommand{\mjmtolx}[1]{}
\newcommand{\mjmtolxx}[1]{}
\newcommand{\mjmtol}[1]{
 \fbox{  
% thi does not ing work right 
\begin{minipage}[t]{\textwidth}
{ \centering{\bf{Thinking outloud}} }
\par   
#1 
\end{minipage} 
}
}

\newcommand{\mjmpicture}[3]
{
\begin{figure}[H]
{ \includegraphics[height=3in,width=4in]{keep/#1} }
\caption{#2}
\label{fig:#3}
\end{figure}
} % mjmpicture


\newcommand{\mjmaside}[1]{
 \fbox{  
% thi does not ing work right 
\begin{minipage}[t]{\textwidth}
{ \centering{\bf{Aside: }} }
\par   
#1 
\end{minipage} 
}
}




\newcommand{\mjmgraphics}[1]{#1 }
\newcommand{\mjmfullplot}[1]{\includegraphics[width=\fullfigwidth]{#1}}
%\newcommand{\mjmincludeplot}[1]{\includegraphics[width=\fullfigwidth]{#1}}
%\newcommand{\mjmincludeplot}[1]{\includegraphics[width=\subfigwidth]{#1}}
\newcommand{\mjmincludeplot}[1]{\includegraphics[height=3in,width=\fullfigwidth]{#1}}

% include here as likely to be doc specific 
%\newcommand{\mjmreffig}[1]{Fig. \ref{#1}}
\newcommand{\mjmreffig}[1]{Fig. \ref{fig:#1}}
\newcommand{\mjmreftab}[1]{Table  \ref{tab:#1}}
\newcommand{\mjmrefeqn}[1]{Eqn  \ref{eqn:#1}}


%cp yyy2.pdf ~/d/latex/keep/pp20171124biotin.pdf
\newcommand{\mjmdatedplot}[1] 
{ \includegraphics[height=3in,width=\fullfigwidth]{keep/pp20171124#1} }
% right now there are too many figs to save for the end apaprently
% hard limit is 36
\newcommand{\mjmbeginfigure}{\begin{figure}[H] }
%\newcommand{\mjmbeginfigure}{\begin{figure}[p] }
\newcommand{\mjmfigure}[1]{
%\begin{figure}[H]
\mjmbeginfigure

#1 

\end{figure}
}
%\extrafloats{100}

\newcommand{\mjmlisting}[1]
{
\begin{lstlisting} 
#1 
\end{lstlisting}
}

\newcommand{\mjmold}[1]{ } 



\newcommand{\mjmeqn}[1]{\begin{equation} #1 \end{equation}  } 





\begin{document}

\title{\mjmtitle}
\author         {Mike Marchywka}
\email          {\mjmemail}
\thanks{ to cite  or credit this work, see bibtex in \ref{appendix:citing} } 
\date{\today}
\affiliation{\mjmaddr}

\mjmblackboxno{Release Notes  xxxx-xx-xx : }{
not ready at all

The AD literature seems to go from miracle to failed miracle almost
daily with little attempt to integrate new results with old  observations.
There are now a lot of data poitns to put  together with cause and effect 
like any TV detective would do or as described in fables like the 
blind men and the elephant even though apparently  Natalie Merchant set
it to music. 
This work tries to fix that 
and finds that all these things are logically explained by 
well known issues such as age related digestive defects.
Suggestions are further supported by in-house work on 
optimization of dog diets demonstrating apparent safety
and utility of out of favor vitamins that could be impacted
by changes in stomach chemistry. 
Another problem may be politics. One issue may be related
to tyrosine metabolism and increased relative
abundance of analogs capable of becoming toxic. This immediately
jumps out as a contributor to race based AD differences not
just social aspects that are touted as they are  politically correct.
To be sure, constituitive tyrosine metabolizers may have adaptive
responses which support this flux with no net "bad" effects
elsewhere and theere may turn out not to be any anwyay. 
In any case, its an important issue and like any other
law of nature will not yield to human desire and the only way
for people to benefit is follow the data. 
If you want to fool society, appease angry people,
 instead of solving problems and eliminating run on sentences 
 for everyone turn back now to your safe space...

\mjmreleasewarning

\mjmstatuswarn

\mjmwarnfeed

\mjmwarnme

\mjmwarntopic

\mjmexplainbib


}

\begin{abstract}
Effective treatments for Alzheimer's Disease(AD) have been difficult
to create. Much effort has been directed at amyloid beta removal
with minimal clinical benefits demonstrated.
Several alternative approaches have become popular including
the infection hypothesis. Since so many pathogens appear to 
causes AD an age related vulnerability may be a better focus. 
Over various time spans, a series of headlines
has emerged claiming benefits for various nutrients or supplements
such as thiamine, choline, lithium, or NAD+. 
However, none appear to generate robust recovery. 
Taking  these results and other  observations
together, similar to the fable of the blind men
grasping different part of an elephant, a plausible unifying
theme is nutrient deficiency due to age related digestive defects.
Isolated profound nutrient deficiencies will not likely exist
so obvious attribution of disease to single molecular entity is not
a good starting point.  
Note that this goes beyond failure to absorb nutrients but also
the creation of age related toxins, possbly methanol for example,
that may need to be mitigated. 
Adaptive and malapative 
responses may make such an  initial cause difficult to discern.
This work motivates "meal engineering" as an intervention
strategy that may be able to correct many of these problems.  
If this analysis accurately
captures cause and effect in the real disease, correction
or prevention may be fairly easy with well known small 
nutrient molecules or addition of acid or chloride to meals.
Individual responses to nutrient details will 
appear to make each case different but still possibly treatable
with similar simple remedies. 
Indeed, there has been some limited success with dietary interventions
but these fail to be designed around likely age related causes
although they may coincidentally mitigate some of these problems. 
It may not be possible get all required nutrients from food once
damage to digestive system has occured and supplements or other entities
may be required.
Cations such as metals, protein bound nutrients, amino acids, 
and lipophilic nutrients
would likely be the first suspects. In particular, its likely that
vitamin K and copper, along with more accepted components such
as amino acids and SMVT substrates all need to be included
to replace the younger absorption profile.  However, brute
force supplementation without regard to formulation
may not work due to chemical and metabolic interactions.  
Two important amino acids with solubility issues in particular,
tyrosine and tryptophan, and critical but reactive metals such
as copper may require special attention. There are also possible
political concerns if there is a hesitancy to link tyrosine metabolism
to dementia. However, as always, social influences will not change the part to 
a best solution.  
 

\end{abstract}

\maketitle
\tableofcontents

\vspace{.5in}
Citing information should also be available in machine readable
form in the pdf extended information unless modified by the hosting
web site ,

\vspace{.3in}

\begin{minipage}{\linewidth}
%\input{bibtex2.txt}
%\input{bibtex3.txt}
\mjmshowbib
\end{minipage}



\newpage


\section{Introduction  }
Alzheimer's disease remains an unresolved problem despite
all the money and effort directed at it for the last
several decades. 
Despite decades of development, anti-amyloid products continue
to show questionable net clinical benefiti
\cite{Panza_Dibello_Sardone_Successes_failures_2025}
 and do not always
get FDA approval even with creative trial design
\cite{Forlenza_AlencarPiresBarbosa_What_reasons_2025}
.
This is in spite of the fact that as early as 2002
some had noted too many inconsistent observations of amyloid
or tau to be serious targets
\cite{Mudher_Lovestone_Alzheimer_disease_2002}
.

Others are using AI for precision medicine thinking that
patient heterogeneity is a problem  \cite{WELCHMAN2026100397}.  

\mjmtol{ generally with such a narrow window there is either an
important constraint or it is just chasing noise. }

In response, some are proposing the use of emerging AI
\cite{FUNK2026100402} which may yield results
but often is limited by traning not on raw data but
human commentary and a statistical rather than "model
of reality" based view.

Some are searching for connections between 
apparently unrelated observations \cite{Hein2025}.

Headlines sometimes appear about new scientific or anecdotal
evidence pointing to a new treatment but rarely do these
translate into useful therapies. The thesis of this work
is that many of these results are useful but tend to be
interpretted in isolation and without regard for cause
and effect in real disease.
"Animal models" may mimic some features of real disease but
are predicated on a cause unlikely to be relevant to 
the pathology being treated. By coincidence, sometimes however
results may be transferable but don't seem to add to understanding.


In the search for alternatives to the amyloid hypotheses,
a number of other ideas have been generated. 
A variety of metabolic ideas have been considered.

Genetics have been investigated too with APOE continuing
to get most of the attention. This gene was identified
as important for vitamin K transport to bones as early
as 1996 with apoe4 noted for
association with lower serum K and increased
fracture risk \cite{KOHLMEIER19961192S} and it was noted response
to supplementation occurs over months.  
However,  a later work found "superior vitamin K status is associated 
with the apoE4 genotype in healthy older individuals 
from China and the UK" 
\cite{Yan_Zhou_Nigdikar_Effect_apolipoprotein_2005}
.
APOE is associated with many vitamins including vitamin K.


A 2025 article found ADAMTS2 to be one of the  genes
robustly  increased  in black and European Alzheimer's brains 
\cite{Teich_Tobunluepop_Troncoso_Novel_differentially_expressed_genes_2025}.
Essentially this translates into pervasive "weak structural proteins"
as described below. 






As no particularly good suspect has emerged, it may be worth
lookin at all the data and trying to guess at age related
vulnerabilities that cause more infections to lead to dementia
and classic symptoms such as amyloid build ups. Nutritional 
defects make some sense as frailty, more than age per se,
may be a correlate of age related diseases.

Works cataloging the utility of vitamins or derivatives
for AD exist \cite{Varma2023} but generally treat these things
as "one at a time" approaches. 
Some mixes such as a mitrochondria nutrient mix 
\cite{Liu_Ames_Reducing_mitochondrial_decay_with_}
have been considered.

In a prior work \cite{marchywka-MJM-2023-008-0.30rg}
I interpreted a brain microbiome result to describe some
host factors that would explain differences in the most and
least abundant organisms in AD and healthy brains post mortem.
While the existence of a brain microbiome is controversial
due to the expected low absolute abudance and opportunity
for contamination, the analysis is somewhat robust in that
contamination from other parts of the patient should
still reflect differential body nutrient levels. 
One argument against contamination was similarity
to an unrelated work on uterine microbiome and possible synergistic
role of these organisms but still a lot of coincidences can occur.
This work generally pointed to deficiencies in lipophilic amino
acids but also suggested additional methanol in the AD patients.
One likely source for this is the GI tract due to changes in microbiome
or initially changes in chemistry. Other toxins could associate
with methanol production too and a good strategy may be to
restore GI chemistry back to youthful state.  

The GLP-1 data may also motivate an interest. GLP-1's apparently
are causing non-specific nutrient deficiencies maybe similar to
problems common in old age. Clinical trials showed either no
benefit in AD or a positive trial
\cite{Edison_Femminella_Ritchie_Liraglutide_mild_2025}
 appeared to be cherry picked.
Meanwhile anecdotes about an aged appearance ( " Ozempic face" )
and brain fog persist. There are a lot of lessons here but the
immediate thing of relevance is that nutrient limitation that
is can be "plausibly denied" appears to generate some symptoms
of aging and more to the point frailty.

Evidence pointing towards a cognitive benefit for GLP-1's may
be a bit misleading. In one case, they were run against
sulfonylureas which have known relevant side effects.
In other cases, words suggesting a cognitive benefit in actuality
mean that some risk factor has been reduced without 
clear evidence of real clinical benefit against dementia.



Women and blacks appear to be at higher risk of AD 
than European males. 
Factors unrelated to the disease per se may confound results
but that is the case with even politically correct association
studies. 

One obvious issue with skin pigment is tryosine metabolism.
Both the depltion of tyrosine and generation of toxic analogs
could be considered. It is possible that constituitive
skin pigmentation (CSP)  evolved with adaptive mechanisms that
may be instructive to investigate.   While CSP may predispose
to various maladies, it may also be easily treated if
existing adpatations are not already active. Indeed,
speculation on beneficial tyrosine analogs can not be dismissed.
 

Tyrosine anecdotes around menopause may be one indicator.
While not considered essential as it can be derived from 
phenyalanine, it is possible that overall supply of the precursor
and regulatory systems reduce tyrosine production pathologically.
This is most likely during times of "stress" and shortages.
Interestingly in connection with GLP-1 there is a tyrosine
dipeptide apparently acting as a signalling molecule
Amino acids such as tyrosine may have limited availability
in cell cultures and dipeptides have been explored 
\cite{Kang_Mullen_Miranda_Utilization_tyrosine_2012} for many years now.
Also of interest is the need to supplement tyrosine as
a dipeptide to Chinese hamseter ovary cells in bioreactors
\cite{PMC7481768}
with work showing possible pickiness of cells for uptake
although translation to human GI tract remains to be done.
Similar problems have been noted for tyrosine usage in humans
from either IV injection or parenteral feeding \cite{MAHER1990685}.
Tyrosine-tyrosine peptide is considered a hormone and may
be responsible for anorexia in urea cycle disorder patients
\cite{PMC3336020}.
Indeed GLP-1 is often considered with PYY for understanding obesity
signalling \cite{PMC3286726}. While its unclear if PYY can replace
the GLP-1 drugs, if supplementation is attempted it will benefit
a lot from recognizing the issues explored here. Dipeptides or
reproduction of "best' GI chemistry may be more important than
simply adding tyrosine to diet.


Copper has a lot of motivation behind it but its rich chemistry
may make it one of the more difficult to study.

For whatever reason, vitamin K is often ignored even though 
interactions with APOE and appearance in "healthy" foods
motivates at least an association with health.


Choline has also made recent headlines

adding to a long history for it 
\cite{Velazquez_Ferreira_Knowles_Lifelong_choline_supplementation_ameliorates_2019}
\cite{10.7554/eLife.89889}
and related pathways 
\cite{FU2004258} including cholinesterase inhibitors
\cite{Dementia_CognitiveImprovementGroup_Cholinesterase_inhibitors_Alzheimer_2016}.

Biotin and the SMVT substrates
\cite{Sang_Philbert_Hartland_Coenzyme_Dependent_Tricarboxylic_2022}
 have also come up in AD work
and dementia more generally
\cite{Scholefield_Church_Xu_Localized_Pantothenic_Acid_Vitamin_2024}
\cite{Cooper_P3400_Biotin_deficiency_2008}
.
Biotin is critical for many processes and can be obtained from
digestion or intestinal microbe production
\cite{Karachaliou_Livaniou_Biotin_Homeostasis_Human_2024}.


Thiamine is also being considered \cite{Fessel_Supplemental_thiamine_2021}
and that too is through to have acid sensitive absorption
\cite{PMC12811047}.
A 2012 study did show some benefits from supplementation in the
elderly but due to poor absorption parental dosing was
mentioned 
\cite{quUx1ed1badcharvv7889cLuong_Hoang_Role_Thiamine_2011}.
It seems that many studies on isolated nurients have come to
similar conclusions but not attempted to think through cause
and effect more generally. 
A 1988 study claimed to use a niacinamide placebo 
\cite{Blass_Thiamine_Alzheimer_1988}
which  motivates another problem of an unintentionally active
placebo.
And in fact niacinamide has been discussed as a treatment too
\cite{Klotter_Niacinamide_nicotinamide_Alzheimer_2010}. 


	Lithium has recently been rediscovered 
\cite{Aron_Ngian_Qiu_Lithium_deficiency_2025}
although positive review articles date back to at least 2012
\cite{Forlenza2012} or so 
\cite{Matsunaga_Kishi_Annas_Lithium_Treatment_2015} 
and its not clear what happened over the intervening decade.

In 1971, it was observed that lithium could impact
carbohydrate generally and inositol status in the brain
\cite{ALLISON_STEWART_Reduced_Brain_Inositol_1971}.
In another manuscript considering inositol as a dog supplement
( unpublished result ) 

but as of 2005 it could not be determined if lithium or
inositol was more directly causal to brain changes
that matter in the clinic
\cite{Harwood_Lithium_bipolar_mood_2005}.
This is likely a recurring issue in Alzheimer's and medicine
overall. Associations are difficult to map to an actionable
cause and effect sequence that can be beneficially modified.
It is the main theme of this work that the primary cause
is well removed from these associations and the elephant
is not apparent from the sum of its parts. 
It is likely that both inositl and lithium are 
impacted by some common cause and simply supplementing one
or the other or both can make some improvements but is a deadend
to a cure.


An early proposal for vitamin K deficiency was described in 2000
citing "regulation of sulfotransferase activity and the activity of a growth factor/tyrosine kinase receptor (Gas 6/Axl) " \cite{ALLISON2001151}.

In the last few years many works have explored vitamin K in old
age conditions
\cite{Emekli-Alturfan2023}
\cite{PMC11676630}
including association between vitamin K intake and cognition
\cite{Booth_Shea_Barger_Association_vitamin_2022}.
As bleeding or microbleeding is an issue in AD, its worth noting
that 82 percent of "chronic stroke subjects" were vitamin K intake
deficient and excessive vitamin E may also be a common problem 
\cite{Grimaldi_Cavallaro_Angelis_Vitamin_Properties_2025}.
The above work also describes more direct effects of vitamin K
in AD including interactions with amyloid beta. 
( there are works that concentrate on emerging problems with 
vitamin E supplementation too, for example 
\cite{Kaye_Thomassen_Mashaw_Vitamin_upalpha_Tocopherol_2025}
)



\mjmtol{
}
\section{Common Age-Related Digestion Defects  }
To begin to make sense of the above and look towards
the design of a better intervention, some knowledge
of likely defects and resulting deficiencies needs to be
established. 

By 1992, common themes in impairment reports were emerging.
One review suggested decreased taste and  hypochlorhydria  
with decreased uptake of B-12 and minerals 
but increased absorption of lipids \cite{RUSSELL19921203S}.
 A small 1997  study of independently living people over 65
found about 90 percent had basal pH under 3.5 \cite{Hurwitz_Gastric_Acidity_Older_1997}
although there is no indication of how this compares to the young
or very old.

The pancreas and intestines may also be effected. One review
mentions decreases in calcium, zinc, and magnesium but not copper
in old rats 
\cite{Holt_Intestinal_Malabsorption_2007}.
However, the pancreas and intestines must work with whatever
the stomach produces with the meal. Earlier, I considered that
lack of an early low pH step may produce more idiosyncratic
results as some mineral can irreversibly precipitate
\cite{marchywka-MJM-2024-010-0.01rg}.

Around 2004, it was thought that reserve capacity was sufficient
although some reduced absorption of vitamins 
such as A,D,K, and B6 but tended to minimize significance
for treating aging \cite{Dryden2004}.
One problem however is that even today sufficient intake and
uptake may not be clear and the impact of multiple "low
levels" could invoke adpative responses. 


Today however protein deficiency appears to be reasonably well
accepted as an issue with aging and its manifestation as sarcopenia
\cite{Qiu2025}. What mey be less apparent is any changes in quality
such as due to translational infidelity or lack of "other stuff."
Sarcopenia or frailty appear to be highly correlated with
vulnerability to several "age related" conditions
\cite{YE2023102111}.

Further, the impact of drugs common in the elderly on 
nutrient status is becoming recognized
\cite{Bell_Rodrigues_Antoniadou_Update_Drug_2023}.


 

\section{Meal Engineering Considerations }

Based on the above, the goal then is to replace nutrients
and remove toxics that are more common in old age. Return to a
youthful GI tract may be a reasonable goal although it may be
possible to do even better. Conceivably if nutrient uptake can
overcome GI limitations some healing may be observed
making possible a return to normal food for a while.

Nutrients need only be supplied over "relevant" time scales.
This can vary a lot from nutrient to nutrient and allows 
better segregation and rotation than trying to get everything
everyday. Incompatibilities include chemistry and competition
for receptors or enzymes or transporters etc.




Not every meals needs to be the same and in fact currently
with the dogs one meal contains most things while a second
is largely reserved for copper. 

\section{Candidate Ingredients}

Candidates are drawn from prior thoughts on food and
vitamin supplements along with the presumed GI disturbances
common in old age.

Any foods or supplements reputed to improve some age
related condition could be examined. In many cases
the folklore appears to be a useful starting point
as it may rationalize well with the above theory or
be supported by in-house work here with dog feeding. 


Fermented foods meet many of the criteria for obtaining nutrients
with age impaired digestions. They tend to have low pH, and can even
include pH as part of the definition, and essential molecules  such as free
amino acids and vitamin k. 

Olive oil has been an important part of in-house dog diets 
( manuscript in preparation ) as well as being a highlighted
component of the Mediterranean Diet. A variety of possible causal
pathways could be imagined for it but here it is considered
somewhat uniquely, along with lecithin and phosphorous sources,
as an absorption enhancer. 
Its interesting to note, although of unknown significance, that
yeast with a higher amount of membrane oleic acid have higher
ethanol tolerance
\cite{You_Rosenfield_Knipple_Ethanol_Tolerance_2003}.
As part of a meal formulation it may help with uptake of
lipophilic nutrients.

2 related objectives are to improve chloride levels and reduce
pH. In the results cited previously, acid came from citric 
acid and HCl supplement formulations. Formulations that
use HCl as a raw component would require great care and
may be avoided initially.  
Some soft drinks such as diet cola beverages contain phosphoric
acid that may have some benefits. 
Chloride was largely
from potassium chloride and again added HCl.   

\expandableinput{nutrient_table.tex}


\section{Development Paths}
Since many candidate components are GRAS, its possible to outline
informal or "crowd sourced" data from motivated parties such
as caregivers. I have been using MUQED daily to record
dog diets and relate to outcomes. This helped to identify
olive oil as an improtant component although it may have
a long response time if other sources of oleic
acid such as chicken are included in the diet. 
The added workflow just requires a few minutes after each
meal or dosing to enter whatever was given and have it checked
for basic syntax. I use "vi" although more sophisticated or
voice entry may be easy to do. 
The current
implementation of MUQED creates a small vocabulary of
foods, supplements and outcomes as well as a simply syntax
for entering amounts right after or even skeletons before
consumption. As the current implementation is designed
for just "a few" people or dogs, everything is done with
text files locally although collation with a standard vocabulary
could be done with better scaling implementations.   

My in-house results with dogs does suggest some folk lore
or anecdotes may have useful signals but will require more
data to sort out and a MUQED-like approach may fill the void.

Even if this fails to mitigate disease, it may not be a 
frivolous effort especially
if any trends are observed with prescription medicines.


\section{Conclusions}


\section{Supplemental Information}

\subsection{Computer Code}


\begin{lstlisting}


\end{lstlisting}
\section{Bibliography}


\bibliography{\mjmbasename,\mjmaddbio}
\bibliographystyle{plainurl}


%%%%%%%%%%%%%%%%%%%%%%%%%%%%%%%%%%%%%%%%%%%%%%%%%%%%%%%%%%%%%%%%%%%%%%%%%%%%%
\begin{acknowledgments} 

% \input{generalack.tex}
\begin{enumerate}
\item Pubmed eutils facilities and the basic research it provides. 
\item Free software including Linux, R, LaTex  etc.
\item Thanks everyone who contributed incidental support. 
\end{enumerate}

\end{acknowledgments}

%%%%%%%%%%%%%%%%%%%%%%%%%%%%%%%%%%%%%%%%%%%%%%%%%%%%%%%%%%%%%%%%%%%%%%%%%%%%%
\clearpage
\appendix

\begin{mdpicomment}

\section{ Statement of Conflicts }
 No specific funding was used in this effort and there are no relationships
with others that could create a conflict of interest. I would like to develop
these ideas further and have obvious bias towards making them appear 
successful. Barbara Cade, the dog owner, has worked in the pet food industry
but this does not likely create a conflict. We have no interest in the
makers of any of the products named in this work.  

\end{mdpicomment}

\begin{mdpicomment}
\section{About the Authors and Facility}
This work was performed at a dog rescue run by Barbara Cade and
housed in rural Georgia.  The author of this report 
,Mike Marchywka,
has a background in electrical engineering and 
has done extensive research using free online literature sources.  
I hope to find additional people interested in critically 
examining the results and verify that they can be reproduced
effectively to treat other dogs.

\begin{comment}
\begin{figure}[htb] 
\centering
\mjmed{ picture commented out to save space in drafts...  } 
%\includegraphics[width=\picwidth]{me_on_brick.jpg}
\caption{ 
 }
\end{figure}

\end{comment}


\section{Symbols, Abbreviations and Colloquialisms}

\begin{comment}
% grep "[A-Z][A-Z]" paradox.tex | sed -e 's/[^A-Z]/\n/g' | grep "[A-Z]" | sort | uniq -c
% cat  paradox.tex | sed -e 's/  */\n/g' | grep "[A-Z][A-Z]"  | grep -v "[^A-Z]" | sort | uniq  |awk '{print $0" &   \\\\"; }'
\end{comment}


%\abbreviations{The following abbreviations are used in this manuscript:\\
%\begin{table}
\noindent
\begin{tabular}{@{}ll}
%SMVT & Sodium dependent Multi-Vitamin Transporter\\
TERM & definition and meaning   \\
\hline
%TLA & Three letter acronym\\
%LD & linear dichroism
\end{tabular} % }
%\end{table}

% https://tex.stackexchange.com/questions/5957/bibtex-entry-for-white-papers-and-technical-reports

\section{General caveats and disclaimer }
\label{appendix:caveats}

%\input{disclaimer-informal.tex}
\input{disclaimer-gaslight.tex}
\input{disclaimer-status.tex}

\section{Citing this as a tech report or white paper }
\label{appendix:citing}

Note: This is mostly manually entered and not assured to be error free.

This is tech report \mjmtrno. 

\begin{table}[H] \centering
\begin{tabular}{r|r|c|r}
Version & Date & Comments  &  \\
0.01 & \mjmmakedate  &  Create from empty.tex template  &  \\
-  & \today & version  \mjmversion { }   \mjmtrno  &  \\
1.0 & 20xx-xx-xx & First revision for distribution &  \\
\end{tabular}
\end{table}


Released versions,

build script needs to include empty releases.tex
\begin{table}[H] \centering
\begin{tabular}{|r|r|l|}
Version & Date & URL    \\
\hline
\expandableinput{releases.tex}
%\input{releases.tex}
\hline
\end{tabular}
\end{table}





% 2020-11-30 keep on same page 
%\input{bibtex2.txt}

\begin{minipage}{\linewidth}
%\input{bibtex2.txt}
%\input{bibtex3.txt}
\mjmshowbib
\end{minipage}




\begin{comment}

\end{comment}
\vspace{1cm}
Supporting files. Note that some dates,sizes, and md5's will change as this is
rebuilt.

This really needs to include the data analysis code 
but right now it is auto generated picking up things from prior
build in many cases 
\lstinputlisting{\mjmbasename.bundle_checksums}
\end{mdpicomment}
\end{document}
